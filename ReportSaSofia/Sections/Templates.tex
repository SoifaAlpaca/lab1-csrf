\section{Template}

\begin{table}[h]
    \centering
    \caption{Anti-Aliasing Filter Specifications and Achieved Performance}
    \begin{tabularx}{\textwidth}{>{\centering\arraybackslash}X
                               >{\centering\arraybackslash}X
                               >{\centering\arraybackslash}X
                               >{\centering\arraybackslash}X}
        \toprule
        \textbf{Specification} & \textbf{Target} & 
        \textbf{2\textsuperscript{nd}-Order Butterworth} & 
        \textbf{3\textsuperscript{rd}-Order Butterworth} \\
        \midrule
        Pass-band ripple $A_\mathrm{max}$ (dB)   & $\le 0.5$ & 0.5 & 0.5 \\
        \midrule
        Stop-band attenuation $A_\mathrm{min}$ (dB) & $\ge 80$ & 90 & 90 \\
        \midrule
        Pass-band edge $f_p$ (kHz) & 20 & 20 & 20 \\
        \midrule
        Stop-band edge $f_s'$ (MHz)\footnotemark[1] & 4.62 & 4.62 & 4.62 \\
        \midrule
        Transition ratio $f_s'/f_p$ & 231 & 231 & 231 \\
        \midrule
        Filter order $N$ & — & 2 (chosen) & 3 (strict) \\
        \midrule
        Theoretical in-band group delay\footnotemark[2] ($\mu$s) & — & 7.9 & 11.8 \\
        \bottomrule
    \end{tabularx}
    \label{tab:aa_filter_specs}
\end{table}

\footnotetext[1]{First stop-band edge equals $f_s - f_p$, where $f_s$ is the modulator sampling frequency (\SI{4.64}{\mega\hertz}).}
\footnotetext[2]{Approximate group delay evaluated at \(\omega_p\) for a Butterworth LPF: \(\tau_g \approx N/(2\pi f_p)\).}
%TODO FOTOS LADO A LADO
\begin{figure}[H]
    \centering
    \includegraphics*[scale = 0.05]{Images/NovaFctHor.png}
    \caption{Logo da Nova FCT}
    \label{wrap-fig:1}
\end{figure}

%Equation System
\begin{equation}
    \begin{cases}
    
        R( 283,15 ) = 1,998\cdot 10^4 ~\Omega \\
        R( 298,15 ) = 10^4 ~\Omega\\
        R( 313,15 ) = 0,5282 \cdot 10^4 ~\Omega\\
    
    \end{cases}
    \Leftrightarrow
    \begin{cases}
        A = 1,3092 \cdot 10^{-3}\\
        B = 2,1439 \cdot 10^{-4}\\
        C = 9,6600 \cdot 10^{-8}\\
    
    \end{cases}
\end{equation}

\begin{lstlisting}[language=Matlab, caption=Matlab code example]
    Fdz
    printf('Polos: ');
    PlFdz
    %figure(3);
    pzmap(Fdz);
    %figure(4);
    step(Fdz);
\end{lstlisting}

\begin{itemize}
    \item item 1
    
    ...
    \item item n 
\end{itemize}

\begin{enumerate}
    \item Butterworth

    \item Chebyshev
    
    \item Elliptic
    
    \item Bessel
    
    In the application in study, the group delay is a critical factor because the ECG signal is a time-domain signal, and the phase distortion can lead to a misinterpretation of the signal. So it is safe to say that the Bessel filter is the best choice for this application.
\end{enumerate}

\begin{figure}[H]   
\begin{centering}
    \begin{tikzpicture}[node distance=2cm]

        % Blocks
        \node (ntc) [block] {NTC Resistance};
        \node (ohms) [block, right of=ntc, xshift=2.5cm] {Ohms to Voltage};
        \node (mcu) [block, right of=ohms, xshift=1.5cm] {MCU};
        \node (temp) [block, right of=mcu, xshift=1cm] {Temperature};
    
        % Arrows
        \draw [arrow] (ntc) -- (ohms);
        \draw [arrow] (ohms) -- (mcu);
        \draw [arrow] (mcu) -- (temp);
    
    \end{tikzpicture}
    
    \caption{ NTC's block diagram }
    \label{fig:NTCBlock}

\end{centering}
\end{figure}


Referece like this\textsuperscript{\cite{ESP32-datasheet}}
% Add citation commands where necessary
% Example: \cite{reference_label}