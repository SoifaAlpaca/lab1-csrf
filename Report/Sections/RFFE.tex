\section{RFFE experiment setup, results, and analysis}

In this section, the experimental Radio-Frequency Front End circuit for an FM Receiver will be presented. The complete Block Diagram of the FM Receiver can be seen in Figure \ref{fig:RFFE_Block}.

\begin{figure}[H]
    \centering
    \includegraphics*[width=0.7\textwidth]{Images/RFFE_Block.png}
    \caption{FM Receiver RFFE Block Diagram}
    \label{fig:RFFE_Block}
\end{figure}

This receiver is divided in two parts, first, the Analog circuit, consisting of the LNA, Mixer, Local Oscillator and finally, an ADC to allow the remaining signal processing to be made by the GNU-Radio.

\subsection{RFFE circuit setup}

To implement the Analog RFFE circuit present in Figure \ref{fig:RFFE_Circuit}, it was used one PCB mounted LNA, the ADC831 Mixer and for the Local Oscillator generator, it was used the ADF4351 PLL Synthesizer, this component generates a stable signal with the same phase of the modulated signal received in the antenna. For the ADC and the connection between the Analog circuit and the GNU-Radio Interface, the ADALM2000 was used. 

\begin{figure}[H]
    \centering
    \includegraphics*[width=0.7\textwidth]{Images/RFFE_Circuit.jpeg}
    \caption{FM Receiver RFFE Analog circuit}
    \label{fig:RFFE_Circuit}
\end{figure}

After connecting the circuit to the supply, the PLL starts the phase locking sequence, and after a few moments, the ADF4351 signals the locking of the phase and generates the correct LO signal, with the Fourier Transform displayed in Figure \ref{fig:RFFE_LO}.

\begin{figure}[H]
    \centering
    \includegraphics*[width=0.7\textwidth]{Images/RRFE_LO.jpeg}
    \caption{LO spectrum generated by the PLL}
    \label{fig:RFFE_LO}
\end{figure}

The resulting signal is then a Square wave with a fundamental frequency of approximately $90MHz$.

Passing then to the GNU-Radio software, the full elements are presented in Figure \ref{fig:RFFE_GNU}.

\begin{figure}[H]
    \centering
    \includegraphics*[width=0.7\textwidth]{Images/RFFE_GNU.jpeg}
    \caption{GNU-Radio FM Receiver}
    \label{fig:RFFE_GNU}
\end{figure}

In this part, it is present, first, the ADALM2000 interface, thus enabling connectivity between both sections of the FM Receiver, followed another Mixer, responsible for enabling synchronization with a specific carrier frequency different form the one present in the LO used previously, this allows a receiver with different receiving carrier frequencies, resulting in an FM Radio Receiver. Finally, the Wide-Band FM demodulator extracts the original data from the received signal.

\subsection{Results}

After tuning the Mixer in the GNU-Radio, it is possible to synchronize the receiver with any Radio Station transmitting FM signal. The results of the FM Receiver can be seen in Figure \ref{fig:RFFE_Result}.

\begin{figure}[H]
    \centering
    \includegraphics*[width=0.7\textwidth]{Images/RFFE_Result.jpeg}
    \caption{FM Receiver result}
    \label{fig:RFFE_Result}
\end{figure}

\textcolor{red}{não sei mais o que dizer}