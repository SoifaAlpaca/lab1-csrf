\section{Introduction}

This laboratory work provides a comprehensive study of Radio-Frequency (RF) communication systems, bridging the gap between high-level system simulation, theoretical circuit analysis, and practical hardware implementation.

The primary objectives of this lab were to:
\begin{itemize}
    \item Understand and simulate a basic AM communication link using GNU Radio.
    \item Model and analyze the impact of non-ideal components in a digital (QPSK) communication system, specifically the non-linearities of amplifiers.
    \item Perform the complete design, simulation, and practical verification of a passive RF circuit (an L-Match network) using a Vector Network Analyzer (VNA).
    \item Assemble, test, and characterize a practical Radio-Frequency Front-End (RFFE) low-IF receiver, capable of demodulating real-world FM radio signals.
    \item Compare theoretical calculations, simulation results, and practical measurements to validate the design process.
\end{itemize}

The report is structured to follow these objectives, beginning with the AM communication link simulation, followed by the analysis of Digital Communication and non-ideal effects in digital modulation, the SPICE simulation of the AM communication circuit, the implementation and measurement of an L-match network using a VNA, and concluding with the practical implementation and testing of the RFFE low-IF receiver.

